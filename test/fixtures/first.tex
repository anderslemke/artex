% This percent indicates a comment.
% This is a very simple latex article that introduces the way 
% equations are typeset. Do this in GNU/Linux:
%
% latex first.tex
% latex first.tex
% xdvi first.dvi
% dvips -o first.ps first.dvi
% gv first.ps
% lpr first.ps
% pdflatex first.tex
% okular first.pdf
\documentclass[12pt]{article}
\title{First \LaTeX} 
\author{Joe Student}
\date{\today}

\begin{document}

\maketitle

\abstract{This is a very simple example of using \LaTeX\ for typesetting.
The procedure for typesetting equations is introduced.}
\begin{sdsd}
\end{bar}

\section{A few equations}
Equations can be typeset inline like so: $\vec{F}=m\vec{a}$ .
Equations can also be separated form the text: $$ \vec{F}=m\vec{a} \ . $$
Notice the punctuation, the ``.'', had to be included with the equation. 
Equations can also have a number assigned and a label attached,
as in the following:
\begin{equation}
\frac{D\vec{\omega}}{Dt} =
( \vec{\omega} + \vec{\Omega}) \cdot \nabla  \vec{U}
+ \frac{1}{\rho^2} \nabla \rho \times \nabla p 
+ \nu \nabla^2  \vec{\omega} 
\label{vorteq}
\end{equation}
The vorticity equation (\ref{vorteq}) is referenced by label,
not by number.  The numbers may change as the document grows and
more equations are added.

New paragraphs are indicated with a blank line in the source code.

\section{The End}

Further examples of \LaTeX\ can be found at {\tt it.metr.ou.edu}

\end{document}
